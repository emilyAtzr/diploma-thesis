\begin{flushleft}
	
	\subsection*{Abstract}

	\subsubsection*{Development of a Mobile, Cloud-Based Web Frontend for an Existing Medical Software in the CGM MAXX LITE Project:}

    The goal of this thesis is to develop a mobile prototype of the existing medical software CGM MAXX, which can be used on smartphones and tablets to optimize workflows in medical practices.

    The thesis begins by exploring the software architecture, highlighting the challenges and their significance. This is followed by a requirements analysis that serves as the basis for selecting an architectural style. A selection of relevant cloud-based architectures is described. Additionally, DevOps practices and Infrastructure as Code (IaC) are considered to enable efficient deployment and maintenance of software in the cloud.
    
    The second chapter focuses on the design of a user-friendly web interface. It introduces the fundamental principles of User Experience (UX) and explains methods to test and implement these principles in practice. The process from concept development to the final app is then described, including the practical implementation and design approaches used.
    
    The third chapter is dedicated to the quality of the source code and discusses methods for ensuring the understandability of the code to facilitate future development. The importance of static code analysis and code reviews is also highlighted to identify errors early and improve the maintainability of the code. Another topic is refactoring, which aims to optimize the code, making it structured and scalable, so that the project can be easily expanded in the long term.
    
    The final part of the thesis explains how the quality of the code can be ensured through testing. The focus here is on Behavior-Driven Development (BDD) and End-to-End (E2E) testing. Using Selenium and Cucumber, the entire user experience is tested by describing scenarios that simulate the system's behavior from the user's perspective. This enables close collaboration between developers and business departments. Additionally, backend testing is addressed, using stubs to simulate components and verify the functionality of the system.

\end{flushleft}