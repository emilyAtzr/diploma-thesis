\begin{flushleft}
	
	\subsection*{Zusammenfassung}

	\subsubsection*{Entwicklung eines mobilen, cloudbasierten Webfrontends für eine bestehende Arztsoftware im Projekt CGM MAXX LITE:}

    Das Ziel dieser Diplomarbeit ist die Entwicklung eines mobilen Prototyps der bestehenden Arztsoftware CGM MAXX, der auf Smartphones und Tablets genutzt werden kann, um Arbeitsabläufe in Arztpraxen zu optimieren.

    Zu Beginn wird die Softwarearchitektur beleuchtet, indem Herausforderungen und deren Bedeutung hervorgehoben werden. Anschließend folgt eine Anforderungsanalyse als Grundlage für die Wahl eines Architekturstils. Dabei wird eine Auswahl relevanter cloudbasierter Architekturen beschrieben.
    Zudem werden DevOps-Praktiken und Infrastructure as Code (IaC) betrachtet, um eine effiziente Bereitstellung und Wartung von Software in der Cloud zu ermöglichen.

    Das zweite Kapitel befasst sich mit der Gestaltung einer benutzerfreundlichen Weboberfläche. Dabei werden die wesentlichen Prinzipien der User Experience (UX) vorgestellt und Methoden erläutert, wie diese in der Praxis getestet und umgesetzt werden können. Im Anschluss wird der Prozess von der Konzeptentwicklung bis zur fertigen App beschrieben, einschließlich der praktischen Umsetzung und der verwendeten Designansätze.

    Das dritte Kapitel widmet sich der Qualität des Quellcodes und behandelt Methoden zur Sicherstellung der Verständlichkeit des Codes, um eine einfache Weiterentwicklung in der Zukunft zu gewährleisten. Zudem wird auf die Bedeutung von statischer Codeanalyse und Code Reviews eingegangen, um Fehler frühzeitig zu identifizieren und die Wartbarkeit des Codes zu verbessern.
    Ein weiteres Thema ist Refactoring, das dazu dient, den Code zu optimieren, ihn strukturiert und skalierbar zu gestalten, sodass eine
    langfristige Erweiterung des Projekts möglich ist.

   Im letzten Teil der Diplomarbeit wird erklärt, wie die Qualität des Codes anhand von Testing sichergestellt werden kann. Dabei liegt der Fokus auf Behavior-Driven Development (BDD) und End-to-End (E2E)-Testing. Mithilfe von Selenium und Cucumber wird die gesamte Benutzererfahrung getestet, indem Szenarien beschrieben werden, die das Verhalten des Systems aus der Sicht des Benutzers simulieren. Dies ermöglicht eine enge Zusammenarbeit zwischen Entwicklern und Fachabteilungen. Zusätzlich wird das Backend-Testing behandelt, wobei Stubs zum Einsatz kommen, um Komponenten zu simulieren und die Funktionalität des Systems zu überprüfen.

\end{flushleft}